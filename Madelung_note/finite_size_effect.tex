\documentclass{article}
\usepackage{amsmath}
\usepackage{mathrsfs}
\usepackage{graphicx}
\usepackage{amssymb}
\usepackage {subcaption}
\usepackage{float}
\usepackage{multirow} %表格
\usepackage{siunitx} %SI单位
\usepackage{makecell} %表格里一对二
\usepackage[flushleft]{threeparttable}
\usepackage[hidelinks]{hyperref}
\title{Notes: The Madelung problem of finite crystals}
\author{Zhao Yihao \\ \textit{zhaoyihao@protonmail.com}}
\begin{document}
	\maketitle
	
	
	The Madelung constant is a dimensionless geometric factor that represents the total Coulomb potential at a specific ion site in a crystal lattice, due to all other ions in the lattice, scaled relative to the nearest-neighbor distance. \par 
	
	\vspace{\baselineskip}
	
	In an infinite crystal, the total Coulomb potential is given by 
	\begin{equation}
		\phi(\mathbf{r}_i) = \sum_{j=1,\, j\ne i} 	u_{\text{pair}}^{\infty}(\mathbf{r}_i - \mathbf{r}_j ) ,
	\end{equation}
	Where the complete electrostatic interaction $u_{\text{pair}}$ is
	\begin{equation}
		u_{\text{pair}}^{\infty}(\mathbf{r}) =  \dfrac{1}{r_{ij}} + \sum_{\mathbf{n}\ne 0}^{\infty} \bigg{(} \dfrac{1}{|\mathbf{r} + \mathbf{n}|} - \dfrac{1}{|\mathbf{n}|}\bigg{)} .
	\end{equation}
	This is a conditionally convergent infinite series. The rule for its summation rearrangement is related to the choice of unit cell and the tiling rule (shape) of the cell for expanding the lattice (detailed discussion can be found in my thesis). \par 
	
	
	According to a specific summation order corresponding to the Ewald method, we obtain:
	\begin{equation}
		u_{\text{pair}}^{tf}(\mathbf{r}) + u_{\text{pair}}^{ib}(\mathbf{r}) \quad : \quad \text{a rearrangement of }  u_{\text{pair}}^{\infty}(\mathbf{r}).
	\end{equation}
	Therefore, the computable expression for an infinite crystal is
	\begin{equation}
		\phi(\mathbf{r}_i) = \sum_{j=1,\, j\ne i} 	u_{\text{pair}}^{tf}(\mathbf{r}_i - \mathbf{r}_j ) + u_{\text{pair}}^{ib}(\mathbf{r}_i - \mathbf{r}_j ).
	\end{equation} 
	\par \vspace{\baselineskip}
	
	When the crystal size is finite, the complete electrostatic interaction in this case is
	\begin{align}
		u_{\text{pair}}^{\mathcal{L}}(\mathbf{r}) = & \dfrac{1}{r_{ij}} + \sum_{\mathbf{n}\ne 0}^{\mathcal{L}} \bigg{(} \dfrac{1}{|\mathbf{r} + \mathbf{n}|} - \dfrac{1}{|\mathbf{n}|}\bigg{)} \notag \\
		=& 	u_{\text{pair}}^{\infty}(\mathbf{r})  - \sum_{\mathbf{n}\notin \mathcal{L}}^{\infty} \bigg{(} \dfrac{1}{|\mathbf{r} + \mathbf{n}|} - \dfrac{1}{|\mathbf{n}|}\bigg{)} .
	\end{align}
	Hence, a feasible expression for the calculation in a finite crystal is
	\begin{equation}
		u_{\text{pair}}^{\mathcal{L}}(\mathbf{r}) = u_{\text{pair}}^{tf}(\mathbf{r}) + u_{\text{pair}}^{ib}(\mathbf{r}) + u_{\text{pair}}^{corr}(\mathbf{r}) \label{eq:core}
	\end{equation}
	\par \vspace{\baselineskip}
	
	Therefore, for a finite crystal, the key lies in solving the finite-size correction term $u$. Performing a Legendre expansion on its general term yields
	\begin{equation}
		\frac{1}{|\mathbf{r} + \mathbf{n}|} - \frac{1}{|\mathbf{n}|}
		= \sum_{k=1}^{\infty} \frac{(-r)^{k}}{|\mathbf{n}|^{k+1}} \, P_{k}(\cos\gamma) = \sum_{k=1}^{\infty} \dfrac{(\mathbf{r}\cdot \nabla_{\mathbf{n}})^k}{k!}\dfrac{1}{n}, \label{eq:expan}
	\end{equation}
	The first expansion is the Legendre expansion, and the second is the Taylor expansion. Their general terms are equivalent.
	In the Legendre polynomial $P_{k}(\cos\theta)$, where $\cos\theta = \mathbf{r} \cdot \mathbf{n}/(r n)$, $P_k$ is an even function of $\mathbf{n}$. For the correction term $u_{\text{corr}}$, the summation over $\mathbf{n}$ runs over a centrosymmetric domain. Consequently, in the Legendre expansion, because the denominator $|\mathbf{n}|^{k+1}$ is an odd function of $n$, only the even-order (i.e., even $k$) terms survive the summation. \par 
	
	Therefore, in the Taylor expansion, only the even-order terms are retained as well. Substituting the general term back into the lattice sum separately.
	\begin{equation}
		\sum_{\mathbf{n} \notin \mathcal{L}}^{\infty}  \dfrac{(\mathbf{r}\cdot \nabla_{\mathbf{n}})^k}{k!}\dfrac{1}{n} = \sum_{\mathbf{n} \ne 0}^{\infty}  \dfrac{(\mathbf{r}\cdot \nabla_{\mathbf{n}})^k}{k!}\dfrac{1}{n}  - \sum_{\mathbf{n} \ne 0}^{\mathcal{L}}  \dfrac{(\mathbf{r}\cdot \nabla_{\mathbf{n}})^k}{k!}\dfrac{1}{n} 
	\end{equation}
	When $r \ll n$, the discrete lattice sum can be approximated by a continuous integral over the volume $\Omega$. To maintain dimensional consistency and account for the lattice-point density, the factor $1/V$ (the inverse unit-cell volume) is introduced. 
	\par 
	
	\vspace{\baselineskip}
	
	Consequently, the second-order term is expressed as:
	\begin{equation}
		\dfrac{1}{2V}\int_{\Omega(\infty)}   (\mathbf{r}\cdot \nabla_{\mathbf{n}})^2\dfrac{1}{n} \, d \mathbf{n} - \dfrac{1}{2V}\int_{\Omega(\mathcal{L})}  (\mathbf{r}\cdot \nabla_{\mathbf{n}})^2\dfrac{1}{n} \, d \mathbf{n}  \label{eq:cha}
	\end{equation}
	where
	\begin{equation}
		(\mathbf{r}\cdot \nabla_{\mathbf{n}})^2 \dfrac{1}{n} =
		\mathbf{r}\cdot \nabla_{\mathbf{n}}\bigg{(}\mathbf{r}\cdot \nabla_{\mathbf{n}} \dfrac{1}{n}\bigg{)} = \nabla_{\mathbf{n}}\cdot \bigg{[} \mathbf{r} \bigg{(} \mathbf{r} \cdot \nabla_{\mathbf{n}} \dfrac{1}{n}\bigg{)}\bigg{]} .
	\end{equation}
	Applying the divergence theorem:
	\begin{equation}
		\dfrac{1}{2V}\oint_{\partial \Omega} \mathbf{r} (\mathbf{r} \cdot \nabla_{\mathbf{n}} \dfrac{1}{n}) \cdot d \mathbf{S} .
	\end{equation}
	
	The integration domain is a crystal composed of unit cells with side lengths $L_x, L_y, L_z$. The size of the finite crystal is given by:
	\begin{equation}
		\mathcal{L}_x = (2p_1+1)L_x , \quad \mathcal{L}_y = (2p_2+1)L_y , \quad \mathcal{L}_z = (2p_3+1)L_z .
	\end{equation}
	Therefore, this surface integral can be split into three pairs of surface integrals, one of which is:
	\begin{equation}
		\dfrac{1}{2V} \int_{-\mathcal{L}_z}^{\mathcal{L}_z} d n_z \, \int_{-\mathcal{L}_y}^{\mathcal{L}_y} d n_y \, \mathbf{e}_x \cdot \bigg{[} \mathbf{r}\bigg{(}\mathbf{r} \cdot \nabla_{\mathbf{n}} \dfrac{1}{n}\bigg{)}\bigg{|}_{n_x = \mathcal{L}_x} 
		- \mathbf{r}\bigg{(}\mathbf{r} \cdot \nabla_{\mathbf{n}} \dfrac{1}{n}\bigg{)}\bigg{|}_{n_x = -\mathcal{L}_x} \bigg{]} . \label{eq:faceint}
	\end{equation}
	Substituting the variables and performing the integration (detailed calculations can be found in my doctoral thesis), we obtain:
	\begin{align}
		\text{eq}(\ref{eq:faceint}) =& \dfrac{-4x^2\mathcal{L}_x^2}{V} \bigg{[} \dfrac{1}{\mathcal{L}_x^2} \arctan \dfrac{T}{\mathcal{L}_x^2} \bigg{|}_{T=\mathcal{L}_x^2}^{T=(\sqrt{3}+2) \mathcal{L}_x^2} \bigg{]} \notag \\
		=& \dfrac{-4 x^2}{V} \arctan (T) \bigg{|}_{1}^{(\sqrt{3}+2)} = -\dfrac{4\pi}{3} x^2
	\end{align}
	The result shows that the integral is independent of the length $\mathcal{L}_x$ (and hence of $p_1$); the same reasoning applies to the other two pairs of surfaces. Therefore, in eq(\ref{eq:cha}), the integral over the infinite domain $\Omega(\infty)$ is equivalent to taking the limit $p_1, p_2, p_3 \to \infty$,  Along this limiting path, the value remains unchanged, and thus the difference between the two terms in eq(\ref{eq:cha}) is identically zero. \par   
	
	\vspace{\baselineskip}
	
	We now proceed to evaluate the fourth-order ($k = 4$) contribution to the finite‑size correction :
	\begin{equation}
		\dfrac{1}{24 V}\int_{\Omega(\infty)}   (\mathbf{r}\cdot \nabla_{\mathbf{n}})^4\dfrac{1}{n} \, d \mathbf{n} - \dfrac{1}{24 V}\int_{\Omega(\mathcal{L})}  (\mathbf{r}\cdot \nabla_{\mathbf{n}})^4\dfrac{1}{n} \, d \mathbf{n}  
	\end{equation}
	To express the fourth-order term $(\mathbf{r} \cdot \nabla_{\mathbf{n}})^4 (1/n)$ in a form suitable for applying the divergence theorem, we first compute the third-order derivative
	\begin{equation}
		(\mathbf{r} \cdot \nabla_{\mathbf{n}})^3 \frac{1}{n}
		= \frac{9 r^2 (\mathbf{r} \cdot \mathbf{n})}{n^5} - \frac{15 (\mathbf{r} \cdot \mathbf{n})^3}{n^7}.
	\end{equation}
	Noting that $\nabla_{\mathbf{n}} \cdot [\mathbf{r} f(\mathbf{n})] = \mathbf{r} \cdot \nabla_{\mathbf{n}} f(\mathbf{n})$ for constant $\mathbf{r}$, we can write the fourth-order term as an outer divergence:
	\begin{equation}
		(\mathbf{r} \cdot \nabla_{\mathbf{n}})^4 \frac{1}{n}
		= 3\,\nabla_{\mathbf{n}} \cdot \Bigg[ \mathbf{r} \bigg( \frac{3 r^2 (\mathbf{r} \cdot \mathbf{n})}{n^5} - \frac{5 (\mathbf{r} \cdot \mathbf{n})^3}{n^7} \bigg) \Bigg].
	\end{equation}
	Applying the divergence theorem:
	\begin{equation}
		\dfrac{1}{8V}\oint_{\partial \Omega} \Bigg[ \mathbf{r} \bigg( \frac{3 r^2 (\mathbf{r} \cdot \mathbf{n})}{n^5} - \frac{5 (\mathbf{r} \cdot \mathbf{n})^3}{n^7} \bigg) \Bigg] \cdot d \mathbf{S} .
	\end{equation}
	\par\vspace{\baselineskip}
	
	We now evaluate the surface integral by considering pairs of parallel faces of the rectangular domain $\Omega$. The contribution from the pair of faces normal to the $x$-axis ($n_x = \pm \mathcal{L}_x$) is
	\begin{equation}
		I_{n_x} = \frac{1}{8V} \left[ \int_{n_x=\mathcal{L}_x} f(\mathbf{n}) \, \mathbf{r} \cdot d\mathbf{S} + \int_{n_x=-\mathcal{L}_x} f(\mathbf{n}) \, \mathbf{r} \cdot d\mathbf{S} \right],
	\end{equation}
	where 
	\begin{equation}
		f(\mathbf{n}) = \frac{3 r^2 (\mathbf{r} \cdot \mathbf{n})}{n^5} - \frac{5 (\mathbf{r} \cdot \mathbf{n})^3}{n^7}.
	\end{equation}
	On the $n_x = +\mathcal{L}_x$ face, $d\mathbf{S} = +\mathbf{e}_x \, dn_y dn_z$ and $\mathbf{r} \cdot d\mathbf{S} = x \, dn_y dn_z$; on the $n_x = -\mathcal{L}_x$ face, $d\mathbf{S} = -\mathbf{e}_x \, dn_y dn_z$ and $\mathbf{r} \cdot d\mathbf{S} = -x \, dn_y dn_z$. Therefore,
	\begin{equation}
		I_{n_x} = \frac{x}{8V} \int dn_y \, dn_z \left[ f(\mathcal{L}_x, n_y, n_z) - f(-\mathcal{L}_x, n_y, n_z) \right].
	\end{equation}
	Owing to the symmetry of the integration domain with respect to $n_y$ and $n_z$, only the odd part of $f$ with respect to $n_x$ survives the subtraction. A direct calculation gives
	\begin{equation}
		f(\mathcal{L}_x) - f(-\mathcal{L}_x) = 2 \mathcal{L}_x x \left[ \frac{3 r^2}{(\mathcal{L}_x^2 + n_y^2 + n_z^2)^{5/2}} 
		- \frac{5 \mathcal{L}_x^2 x^2 + 15 (y^2 n_y^2 + z^2 n_z^2)}{(\mathcal{L}_x^2 + n_y^2 + n_z^2)^{7/2}} \right],
	\end{equation}
	where the cross term $2 y z n_y n_z$ has been dropped because its integral over the symmetric domain vanishes. Consequently,
	\begin{equation}
		I_{n_x} = \frac{\mathcal{L}_x x^2}{4V} \int_{-\mathcal{L}_z}^{\mathcal{L}_z} dn_z\, \int_{-\mathcal{L}_y}^{\mathcal{L}_y}  dn_y\,
		\left[ \frac{3 r^2}{(\mathcal{L}_x^2 + n_y^2 + n_z^2)^{5/2}} 
		- \frac{5 \mathcal{L}_x^2 x^2 + 15 (y^2 n_y^2 + z^2 n_z^2)}{(\mathcal{L}_x^2 + n_y^2 + n_z^2)^{7/2}} \right] .
	\end{equation}
	
	
	
	We now specialize to the case of a cube, where $\mathcal{L}_x = \mathcal{L}_y = \mathcal{L}_z = \mathcal{L} = (2p+1)L$.  
	Due to the cubic symmetry, the three pairs of parallel faces give equivalent contributions up to cyclic permutation of the coordinates $(x,y,z)$.  
	We therefore focus on the pair normal to the $x$-axis; the results for the $y$- and $z$-directions follow by symmetry.
	
	Exploiting the symmetry of the integration domain, we may restrict the integration to the first octant and multiply by a factor of $4$:
	\begin{equation}
		I_{n_x} = \frac{\mathcal{L} x^2}{V} \int_{0}^{\mathcal{L}} dn_z \int_{0}^{\mathcal{L}} dn_y\,
		\Bigg[ \frac{3 r^2}{R^{5/2}} 
		- \frac{5\mathcal{L}^2 x^2}{R^{7/2}} 
		- \frac{15 y^2 n_y^2}{R^{7/2}} 
		- \frac{15 z^2 n_z^2}{R^{7/2}} \Bigg],
		\quad R = \mathcal{L}^2 + n_y^2 + n_z^2 .
	\end{equation}
	
	We define three auxiliary integrals:
	\begin{align}
		I_1 =& \int_{0}^{\mathcal{L}} d n_z  \int_{0}^{\mathcal{L}} d n_y \dfrac{3}{(\mathcal{L}^2 + n_y^2 +n_z^2)^{5/2}} \notag \\
		=&\dfrac{1}{\mathcal{L}^3} \int_{0}^{1} d n_z  \int_{0}^{1} d n_y \dfrac{3}{(1 + n_y^2 +n_z^2)^{5/2}} = \dfrac{1}{\mathcal{L}^3} J_5 , 
	\end{align} 
	\begin{align}
		I_2 =& \int_{0}^{\mathcal{L}} d n_z  \int_{0}^{\mathcal{L}} d n_y \dfrac{5}{(\mathcal{L}^2 + n_y^2 +n_z^2)^{7/2}} \notag \\
		=&\dfrac{1}{\mathcal{L}^5} \int_{0}^{1} d n_z  \int_{0}^{1} d n_y \dfrac{5}{(1 + n_y^2 +n_z^2)^{7/2}} = \dfrac{1}{\mathcal{L}^5} J_7 , 
	\end{align} 
	\begin{align}
		I_3 =& \int_{0}^{\mathcal{L}} d n_z  \int_{0}^{\mathcal{L}} d n_y \dfrac{15 n_y^2}{(\mathcal{L}^2 + n_y^2 +n_z^2)^{7/2}} \notag \\
		=&\dfrac{1}{\mathcal{L}^3} \int_{0}^{1} d n_z  \int_{0}^{1} d n_y \dfrac{15 n_y^2}{(1 + n_y^2 +n_z^2)^{7/2}} = \dfrac{1}{\mathcal{L}^3} K_2 .
	\end{align} 
	
	Then
	\begin{equation}
		I_{n_x} = \frac{\mathcal{L} x^2}{V} \big( r^2 I_1 - \mathcal{L}^2 x^2 I_2 - y^2 I_3 - z^2 I_3 \big).
	\end{equation}
	
	Substituting the scaled forms of $I_1$, $I_2$, $I_3$ gives
	\begin{align}
		I_{n_x} =& \frac{\mathcal{L} x^2}{V} 
		\Bigg[ \frac{r^2}{\mathcal{L}^3} J_5 
		- \frac{x^2}{\mathcal{L}^3} J_7 
		- \frac{y^2}{\mathcal{L}^3} K_2 
		- \frac{z^2}{\mathcal{L}^3} K_2 \Bigg] \notag \\
		=& \frac{\mathcal{L} x^2}{V} 
		\Bigg[ \frac{r^2}{\mathcal{L}^3} (J_5 - K_2) 
		- \frac{x^2}{\mathcal{L}^3} (J_7 - K_2) \Bigg].
	\end{align}
	
	The required combinations of the geometric constants are
	\begin{equation}
		J_5 - K_2 = \frac{2}{3\sqrt{3}},\qquad
		J_7 - K_2 = \frac{10}{9\sqrt{3}}.
	\end{equation}
	
	Hence,
	\begin{equation}
		I_{n_x} = \frac{6 r^2 x^2 - 10 x^4}{9\sqrt{3}\mathcal{L}^2 V}.
	\end{equation}
	
	By cyclic symmetry,
	\begin{equation}
		I_{n_y} = \frac{6 r^2 y^2 - 10 y^4}{9\sqrt{3}\mathcal{L}^2 V},\qquad
		I_{n_z} = \frac{6 r^2 z^2 - 10 z^4}{9\sqrt{3}\mathcal{L}^2 V}.
	\end{equation}
	
	Summing the three contributions yields the complete fourth‑order term for the finite domain $\Omega(\mathcal{L})$:
	\begin{equation}
		\frac{1}{24V} \int_{\Omega(\mathcal{L})} (\mathbf{r}\cdot\nabla_{\mathbf{n}})^4 \frac{1}{n} \, d\mathbf{n}
		= \frac{6 r^4 - 10(x^4+y^4+z^4)}{9\sqrt{3}\, (2p+1)^2 L^5},
		\qquad (V = L^3).
	\end{equation}
	
	In the infinite‑crystal limit $p\to\infty$, the same expression tends to zero because of the factor $(2p+1)^{-2}$.  
	Consequently,
	\begin{equation}
		\frac{1}{24V} \int_{\Omega(\infty)} (\mathbf{r}\cdot\nabla_{\mathbf{n}})^4 \frac{1}{n} \, d\mathbf{n}
		= \lim_{p\to\infty} \frac{1}{24V} \int_{\Omega(\mathcal{L})} (\mathbf{r}\cdot\nabla_{\mathbf{n}})^4 \frac{1}{n} \, d\mathbf{n}
		= 0.
	\end{equation}
	
	Thus the finite‑size correction contributed by the fourth‑order term is
	\begin{equation}
		u_{\text{pair}}^{\text{corr}}(\mathbf{r})
		= \frac{6 r^4 - 10(x^4+y^4+z^4)}{9\sqrt{3}\, (2p+1)^2 L^5}.
	\end{equation}
	\par 
	
	\vspace{\baselineskip}
	
    By using eq(\ref{eq:core}), we obtain a fully real-space scheme for computing the Madelung constant, which is mathematically equivalent to the Ewald summation under tin-foil boundary conditions.
    \begin{equation}
	    u_{\text{pair}}^{\mathcal{L}}(\mathbf{r}) \;-\; u_{\text{pair}}^{ib}(\mathbf{r}) \;-\; u_{\text{pair}}^{corr}(\mathbf{r})
	    \;=\; u_{\text{pair}}^{tf}(\mathbf{r}) .
	    \label{eq:finalcompute}
    \end{equation}
    The left-hand side consists of three physically distinct contributions: a direct summation over a finite number of unit cells, an infinite-boundary term, and a finite-size correction.  
    The latter two are closed-form elementary expressions; therefore, only the finite direct sum needs to be evaluated numerically.  
    The scheme converges to the correct result with $\mathcal{O}(p^{-4})$ accuracy—the same as that of the Ewald summation method.
    \par 
    
    \vspace{\baselineskip}
    
    
    The relationship between the multipole order $k$ in the correction term and the convergence accuracy $\mathcal{O}(p^\eta)$ of the Madelung constant calculation can be analyzed as follows. 
    
    For a finite cubic crystal of linear size $n = 2p+1$ unit cells per dimension, the finite‑size correction $\nu_{\text{corr}}$ is obtained by summing the multipole expansion (see Eq.~\eqref{eq:expan}) over all lattice vectors $\mathbf{n}$ outside the finite crystal.  
    Only even-$k$ terms survive due to inversion symmetry.
    
    When the discrete sum is approximated by a continuous integral over the excluded volume, the $k$-th term contributes as
    \begin{align}
    	\nu_{\text{corr}}^{(k)} \sim \int_{\text{excluded region}} \frac{P_k(\cos\theta)}{n^{k+1}} \, dV .
    \end{align}
    In spherical coordinates $dV = n^2\,dn\,d\Omega$, the integrand becomes
    \begin{align}
    	\frac{P_k(\cos\theta)}{n^{k+1}} \cdot n^2\,dn\,d\Omega = P_k(\cos\theta) \, n^{1-k} \, dn\,d\Omega .
    \end{align}
    Here, $n^{1-k}$ arises from the product of the multipole denominator $n^{-(k+1)}$ and the volume‑element factor $n^2$.  
    
    The key step is the radial integration:
    \begin{align}
    	\int_{p}^{\infty} n^{1-k} dn = \left. \frac{n^{2-k}}{2-k} \right|_{p}^{\infty} \sim p^{\,2-k} \qquad (k>2).
    \end{align}
    The exponent $2-k$ shows that the integration over $dn$ effectively contributes a factor of $n^1$ to the power counting.  
    Consequently, for the surviving even‑$k$ terms:
    \begin{align}
    	k=4 &\quad \Rightarrow \quad \nu_{\text{corr}}^{(4)} \sim p^{-2},\\[4pt]
    	k=6 &\quad \Rightarrow \quad \nu_{\text{corr}}^{(6)} \sim p^{-4}.
    \end{align}
    A constant term (independent of $p$) would correspond to $k=2$, but such a term vanishes for cubic crystals due to symmetry—consistent with the above scaling, as $k=2$ would give $p^{0}$ (a constant).  
    Thus, after analytically subtracting the $k=4$ contribution, the remaining dominant error comes from $k=6$, leading to the overall $\mathcal{O}(p^{-4})$ convergence of the explicit correction method.
    
    
    
    
	
\end{document}