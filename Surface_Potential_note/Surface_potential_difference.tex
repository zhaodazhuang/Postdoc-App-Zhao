\documentclass{article}
\usepackage{amsmath}
\usepackage{mathrsfs}
\usepackage{graphicx}
\usepackage{amssymb}
\usepackage {subcaption}
\usepackage{float}
\usepackage{multirow} %表格
\usepackage{siunitx} %SI单位
\usepackage{makecell} %表格里一对二
\usepackage[flushleft]{threeparttable}
\usepackage[hidelinks]{hyperref}
\title{Notes: Surface potential difference and charge delocalization}
\author{Zhao Yihao \\ \textit{zhaoyihao@protonmail.com}}
\date{February 5, 2024}
\begin{document}
	\maketitle
	
	Consider the electric potential generated by two charge densities $\rho_1^e(z')$ and $\rho_2^e(z')$ in the bulk:
	\begin{equation}
		\phi(z) =  \dfrac{1}{4\pi \varepsilon_0} \int_{-\infty}^{\infty} d\mathbf{r}' \, \rho^e(z') \dfrac{1}{|\mathbf{r} - \mathbf{r}'|}.
	\end{equation}
	$\rho_1^e(z)$ and $\rho_2^e(z)$ have the same bulk charge number density $\rho_\text{B}$, but differ in their spatial distribution of charge. Specifically,
	\begin{align}
		\rho_1^e(z) =& |e|\cdot \rho_B(z) \star \delta(z) , \\
		\rho_2^e(z) =& |e|\cdot \rho_B(z) \star G(\mathbf{r}) .
	\end{align}
    $G(\mathbf{r})$ is the standard Gaussian distribution function:
    \begin{equation}
	    G(\mathbf{r}) = \dfrac{1}{\sigma (\sqrt{2\pi})^3}e^{-\dfrac{r^2}{2\sigma^2}}
    \end{equation}
    where $\sigma$ is the standard deviation. 
    The electrostatic potential generated by the Gaussian charge distribution can be transformed:
    \begin{align}
    	\phi(z) =&  \dfrac{|e|}{4\pi \varepsilon_0} \int_{-\infty}^{\infty} d\mathbf{r}' \, \rho_B(z') \star G(\mathbf{r})  \dfrac{1}{|\mathbf{r} - \mathbf{r}'|} \notag \\
    	=& \dfrac{|e|}{4\pi \varepsilon_0} \int_{-\infty}^{\infty} d\mathbf{r}' \, \rho_B(z') \dfrac{\text{erf}(\frac{|\mathbf{r} - \mathbf{r}'|}{\sigma \sqrt{2}})}{|\mathbf{r} - \mathbf{r}'|}
    \end{align}
    \par 
    
    The potential difference generated by two charge densities \(\rho^e(z)\) with the same charge number density but different spatial distributions is:
    \begin{align}
    	\Delta \phi(z) =& \Phi_2(z) - \Phi_1(z)  \notag \\
    	=& -\dfrac{|e|}{4\pi \varepsilon_0}  \int_{-\infty}^{\infty} d\mathbf{r}' \, \rho_B(z')  \dfrac{\text{erfc}(\frac{|\mathbf{r} - \mathbf{r}'|}{\sigma \sqrt{2}})}{|\mathbf{r} - \mathbf{r}'|} 
    \end{align}
    then
    \begin{align}
    	\Delta \phi(z) 
    	=& -\dfrac{|e|}{4\pi \varepsilon_0} \int_{-\infty}^{\infty} dz' \,  \rho_B(z') \int_{-\infty}^{\infty} dx' \,  \int_{-\infty}^{\infty} dy' \,  \dfrac{\text{erfc}(\frac{\sqrt{x'^2 + y'^2 + (z-z')^2}}{\sigma \sqrt{2}})}{\sqrt{x'^2 + y'^2 + (z-z')^2}} 
    \end{align}
    Applying a polar coordinate transformation,
     \begin{align}
    	\Delta \phi(z) 
    	=& -\dfrac{|e|}{4\pi \varepsilon_0} \int_{-\infty}^{\infty} dz' \,  \rho_B(z') \int_{0}^{2\pi} d\varphi \int_{0}^{\infty} da \, a \cdot   \dfrac{\text{erfc}(\frac{\sqrt{a^2 + (z-z')^2}}{\sigma \sqrt{2}})}{\sqrt{a^2 + (z-z')^2}} \notag \\
    	=& -\dfrac{|e|}{2 \varepsilon_0} \int_{-\infty}^{\infty} dz' \,  \rho_B(z') \int_{0}^{\infty} da \, a \cdot   \dfrac{\text{erfc}(\frac{\sqrt{a^2 + (z-z')^2}}{\sigma \sqrt{2}})}{\sqrt{a^2 + (z-z')^2}} 
    \end{align}
    Let $\tau^2 = a^2 + (z-z')^2$, so that when $a=0$, $\tau = |z-z'|$,
    \begin{align}
    	\Delta \phi(z)= -\dfrac{|e|}{2 \varepsilon_0} \int_{-\infty}^{\infty} dz' \,  \rho_B(z') \int_{|z-z'|}^{\infty} d\tau\,  \text{erfc} \bigg{(} \dfrac{\tau}{\sigma \sqrt{2}} \bigg{)}
    \end{align}
    Performing integration by parts on the inner integral,
    \begin{align}
	\Delta \phi(z)=& -\dfrac{|e|}{2 \varepsilon_0} \int_{-\infty}^{\infty} dz' \,  \rho_B(z') \bigg{[} \text{erfc}\bigg{(} \dfrac{\tau}{\sigma \sqrt{2}}\bigg{)} \tau\bigg{|}_{|z-z'|}^{\infty} - \int_{|z-z'|}^{\infty} \tau \,d \,\text{erfc}\bigg{(} \dfrac{\tau}{\sigma \sqrt{2}}\bigg{)} \bigg{]} \notag \\
	=& -\dfrac{|e|}{2 \varepsilon_0} \int_{-\infty}^{\infty} dz' \,  \rho_B(z') \bigg{[} - |z-z'|\text{erfc}\bigg{(} \dfrac{\tau}{\sigma \sqrt{2}}\bigg{)} - \int_{|z-z'|}^{\infty} \tau \,d \,\text{erfc}\bigg{(} \dfrac{\tau}{\sigma \sqrt{2}}\bigg{)} \bigg{]}
    \end{align}
    where
    \begin{equation}
    	d \,\text{erfc}\bigg{(} \frac{\tau}{\sigma \sqrt{2}}\bigg{)} = \dfrac{d\, \text{erfc}(\dfrac{\tau}{\sigma \sqrt{2}})}{dt} \, \dfrac{dt}{d\tau}\, d \tau 
    \end{equation}
    and 
    \begin{equation}
    	\text{erfc}\bigg{(} \dfrac{\tau}{\sigma \sqrt{2}}\bigg{)} = \dfrac{2}{\sqrt{\pi}} \int_{\frac{\tau}{\sigma \sqrt{2}}}^{\infty} e^{-t^2} \, dt
    \end{equation}
    so
    \begin{align}
    	\dfrac{d\, \text{erfc}(\dfrac{\tau}{\sigma \sqrt{2}})}{dt} \, \dfrac{dt}{d\tau}\, d \tau  =& \dfrac{2}{\sqrt{\pi}} \bigg{|}_{\frac{\tau}{\sigma \sqrt{2}}}^{\infty} \cdot \dfrac{d \frac{\tau}{\sigma \sqrt{2}}}{d \tau } \cdot d \tau \notag \\
    	=& -\dfrac{\sqrt{2}}{\sigma \sqrt{\pi}} e^{-\frac{\tau^2}{2\sigma^2}} \cdot d \tau 
    \end{align}
    therefore
    \begin{align}
    		\Delta \phi(z)=  -\dfrac{|e|}{2 \varepsilon_0} \int_{-\infty}^{\infty} dz' \,  \rho_B(z') \bigg{[} - |z-z'|\text{erfc}\bigg{(} \dfrac{|z-z'|}{\sigma \sqrt{2}}\bigg{)} + \dfrac{\sqrt{2}\sigma}{\sqrt{\pi}} e^{-\frac{z-z'}{2\sigma^2}}\bigg{]}
    \end{align}
    
    Considering the charge number density along the \(z\)-axis in an interfacial system,
    \begin{equation}
    	 \rho_B(z) = 
    	 \begin{cases}
    	 	0 &  z \to \infty \\
    	 	\text{const} & z \to -\infty
    	 \end{cases}
    \end{equation}
   Therefore, on both sides of the gas–liquid interface, the potential in the gas phase satisfies $\phi(+\infty) = 0$, while in the liquid phase:
   \begin{align}
   	\Delta \phi(-\infty) = -\dfrac{|e|}{2 \varepsilon_0} \rho_B \cdot 
   	\bigg{[} -& \int_{-\infty}^{\infty} dz' \, |z - z'| \, \text{erfc}\bigg{(} \dfrac{|z - z'|}{\sigma \sqrt{2}}\bigg{)} \notag \\
   	+& \int_{-\infty}^{\infty} dz' \, \dfrac{\sqrt{2}\sigma}{\sqrt{\pi}} \, e^{-\dfrac{(z - z')^2}{2\sigma^2}} \bigg{]}
   \end{align}
   where
   \begin{align}
   	\int_{-\infty}^{\infty} dz' \, |z - z'| \, \text{erfc}\bigg{(} \dfrac{|z - z'|}{\sigma \sqrt{2}}\bigg{)} 
   	&= 2 \int_{0}^{\infty} dt \, t \, \text{erfc}\bigg( \dfrac{t}{\sigma \sqrt{2}}\bigg) \notag \\
   	&= \dfrac{\sqrt{2}}{\sigma\sqrt{\pi}} \int_{0}^{\infty} dt \, t^2 e^{-\dfrac{t^2}{2\sigma^2}} \notag \\
   	&= \sigma^2
   \end{align}
   and
   \begin{align}
   	\int_{-\infty}^{\infty} dz' \, \dfrac{\sqrt{2}\sigma}{\sqrt{\pi}} e^{-\dfrac{(z - z')^2}{2\sigma^2}}
   	&= 2 \dfrac{\sigma \sqrt{2}}{\sqrt{\pi}} \int_{0}^{\infty} dt \, e^{-\dfrac{t^2}{2\sigma^2}} \notag \\
   	&= 2\sigma^2.
   \end{align}
   Therefore,
   \begin{equation}
   	\Delta \phi(-\infty) = -\dfrac{|e|}{2 \varepsilon_0} \rho_B \cdot \sigma^2.
   \end{equation}
   Consequently, the potential drop across the interface is
   \begin{align}
   	 \chi &= \Delta \phi(-\infty) - \Delta \phi(+\infty) \notag \\
   	&= -\dfrac{|e|}{2 \varepsilon_0} \rho_B \cdot \sigma^2. \label{eq:Deltachi}  
   \end{align}
   The above expression quantifies the contribution to the surface potential difference from charges of a single sign. In a real molecular system, partial charges of both signs exist, and their delocalization contributes additively to the total surface potential. Generalizing the result, the net contribution from charge delocalization is the sum of the contributions from positive and negative charges:
   \begin{equation}
 	\chi_{\text{deloc}} = \dfrac{|e|}{2 \varepsilon_0} \left( \rho_B^{-} \cdot \sigma_{-}^2 - \rho_B^{+} \cdot \sigma_{+}^2 \right). \label{eq:chideloc}
   \end{equation}
    Here, $\rho_B^{+}$ and $\rho_B^{-}$ are the bulk number densities of the positive and negative charges, respectively, and $\sigma_{+}$, $\sigma_{-}$ are their corresponding delocalization widths. The minus sign indicates that the contributions from positive and negative charges have opposite signs.

   This derivation quantitatively demonstrates that, for a fixed interfacial charge-density profile, the surface potential difference is directly governed by the degree of charge delocalization ($\sigma$), as encapsulated in eq(\ref{eq:Deltachi}). Charge delocalization is therefore identified as the primary microscopic determinant of $\Delta \chi$.
   
   \par\vspace{\baselineskip}
  
   The surface potential of water presents a striking contrast: classical molecular models predict a value near $-0.5$ V, whereas \textit{ab initio} calculations yield a significantly positive potential around $+4.0$ V. We attribute this order-of-magnitude and sign discrepancy primarily to the treatment of \textit{charge delocalization}, particularly the spatial distribution of negative charges, which differs fundamentally between the two approaches. This work uncovers and proves this point quantitatively.
  
  \vspace{2\baselineskip} 
  
  \centerline{
  	
  	\begin{threeparttable}
  		\caption{Parameters and calculated surface potential contributions for various water models.\label{tab:water_models}}
  		\begin{tabular}{cccccccc}
  			\hline
  			& SPC/E\tnote{a}& TIP3P\tnote{b} & TIP4P\tnote{c} & SWM4-DP\tnote{d}& SWM4-NDP\tnote{e}& ab initio\tnote{f}& WFc\tnote{f} \\ \hline
  			$q_\text{O}$(e)& -0.8476 & -0.830 &  & -1.77185 & 1.71636 &  +6  & +6 \\
  			$q_\text{D}$(e)&         &        &        &  1.77185 &-1.71636 &    & \\
  			$q_\text{M}$(e)&         &        & -1.040 & -1.10740 & -1.11466 & & \\
  			$d_{\text{OH}}(\si{\angstrom})$& 1.0 & 0.9572 & 0.9572 & 0.9572 & 0.9572& $\langle0.98584\rangle$ & $\langle0.98584\rangle$\\
  			$d_{\text{OM}}(\si{\angstrom})$& & & 0.15 &0.23808 & 0.24034 & & \\
  			$\theta_{\text{HOH}}$(°)& 109.47 & 104.52& 104.52 & 104.52 & 104.52 & $\langle104.146\rangle$ & $\langle104.146\rangle$\\ 
  			$\mu$(e\si{\angstrom}) & 0.48937 & 0.48628 & 0.45332& 0.51133 & 0.51237 & $\langle0.60832\rangle$& $\langle0.60832\rangle$\\ 
  			$\sigma_{+}$ (\si{\angstrom})  & 0.47140 & 0.43703& 0.43703 & 0.30718 & 0.32017 &$\langle0.27241\rangle\!'$ & $\langle0.27116\rangle$ \\ 
  			$\sigma_{-}$ (\si{\angstrom})  & 0 & 0 & 0 & 0.06687 & 0.03333 & $\langle0.48311\rangle\!'$ &$\langle0.24085\rangle$ \\
  			$\chi_{\text{deloc}}$(V)& -0.56966 & -0.47944 &-0.56237 &-0.78038 & -0.74986& 3.85161&-0.37543 \\
  			$\chi_{\text{dipole}}$(V) &$\approx$ 0 & -0.07 & & & & 0.295 & 0.295\\
  			$\chi$(V) & -0.57 & -0.55 & -0.5 & -0.540 & -0.545 & +4.18 & -0.08 \\ \hline
  		\end{tabular}
  		\begin{tablenotes}
  			\item[]$\chi_{\text{deloc}}$ is obtained through eq(\ref{eq:Deltachi}), $\chi_{\text{dipole}}$ is obtained through dynamic sampling.
  			\item[a] Model from Ref.[X]. The bulk density is $1\text{g/cm}^3$.
  			\item[b] Model from Ref.[X]'s model B. The bulk density is $1\text{g/cm}^3$.
  			\item[c] Model and bulk density $0.936\text{g/cm}^3$ from Ref.[X], $\chi$ from Ref.[X].
  			\item[d] Model, $\chi$ and bulk density is $0.997\text{g/cm}^3$ from Ref.[X], $\sigma_{\pm}$ is derived via the $\mu$ in the bulk phase.
  			\item[e] Model, $\chi$ and bulk density is $0.997\text{g/cm}^3$ from Ref.[X], $\sigma_{\pm}$ is derived via the $\mu$ in the bulk phase.
  			\item[f] The bulk density is $1\text{g/cm}^3$. $\langle\rangle$ represents the average of molecules in the bulk phase, $\langle\rangle\!'$ represents the use of three bulk water molecules placed at $12\si{\angstrom}\times12\si{\angstrom}\times12\si{\angstrom}$ for wave function analysis and averaged.
  		\end{tablenotes}
  	\end{threeparttable} }

  
  \vspace{2\baselineskip} 
  
  Table~\ref{tab:water_models} presents a decisive test of this thesis. 
  For each water model, we compute the delocalization contribution $\chi_{\text{deloc}}$ using the model's bulk charge density $\rho_B$ and its charge delocalization parameters $\sigma_{+}$ and $\sigma_{-}$. The total surface potential $\chi$ (from literature or simulation) is compared against the sum of $\chi_{\text{deloc}}$ and the dipole-orientation contribution $\chi_{\text{dipole}}$. The data reveal a clear trend: for classical point-charge-like models (SPC/E, TIP3P, SWM4-DP, SWM4-NDP), where $\sigma_{-}$ is small, $\chi_{\text{deloc}}$ is negative and accounts for nearly the entire observed $\chi$. The pivotal insight comes from comparing the `ab initio` and `WFc` results. Both describe the same total molecular charge, yet their predicted $\chi_{\text{deloc}}$ differ by over 4 V. This enormous difference is traced directly to their vastly different $\sigma_{-}$ values (see table), quantitatively proving that the representation of charge delocalization is the primary determinant of the surface potential sign and magnitude.
\end{document}